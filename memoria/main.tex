% Plantilla para un Trabajo Fin de Grado de la Universidad de Granada,
% adaptada para el Doble Grado en Ingeniería Informática y Matemáticas.
%
%  Autor: Mario Román.
%  Licencia: GNU GPLv2.
%
% Esta plantilla es una adaptación al castellano de la plantilla
% classicthesis de André Miede, que puede obtenerse en:
%  https://ctan.org/tex-archive/macros/latex/contrib/classicthesis?lang=en
% La plantilla original se licencia en GNU GPLv2.
%
% Esta plantilla usa símbolos de la Universidad de Granada sujetos a la normativa
% de identidad visual corporativa, que puede encontrarse en:
% http://secretariageneral.ugr.es/pages/ivc/normativa
%
% La compilación se realiza con las siguientes instrucciones:
%   pdflatex --shell-escape main.tex
%   bibtex main
%   pdflatex --shell-escape main.tex
%   pdflatex --shell-escape main.tex

% Opciones del tipo de documento
\documentclass[oneside,openright,titlepage,numbers=noenddot,openany,headinclude,footinclude=true,
cleardoublepage=empty,abstractoff,BCOR=5mm,paper=a4,fontsize=12pt,main=spanish]{scrreprt}

% Paquetes de latex que se cargan al inicio. Cubren la entrada de
% texto, gráficos, código fuente y símbolos.
\usepackage[utf8]{inputenc}
\usepackage[T1]{fontenc}
\usepackage{fixltx2e}
\usepackage{graphicx} % Inclusión de imágenes.
\usepackage{grffile}  % Distintos formatos para imágenes.
\usepackage{longtable} % Tablas multipágina.
\usepackage{wrapfig} % Coloca texto alrededor de una figura.
\usepackage{rotating}
\usepackage[normalem]{ulem}
\usepackage{amsmath}
\usepackage{textcomp}
\usepackage{amssymb}
\usepackage{capt-of}
\usepackage[colorlinks=true]{hyperref}
\usepackage{tikz} % Diagramas conmutativos.
\usepackage{minted} % Código fuente.
\usepackage[T1]{fontenc}
\usepackage{natbib}

% Plantilla classicthesis
\usepackage[beramono,eulerchapternumbers,linedheaders,parts,a5paper,dottedtoc,
manychapters,pdfspacing]{classicthesis}

% Geometría y espaciado de párrafos.
\setcounter{secnumdepth}{0}
\usepackage{enumitem}
\setitemize{noitemsep,topsep=0pt,parsep=0pt,partopsep=0pt}
\setlist[enumerate]{topsep=0pt,itemsep=-1ex,partopsep=1ex,parsep=1ex}
\usepackage[top=1in, bottom=1.5in, left=1in, right=1in]{geometry}
\setlength\itemsep{0em}
\setlength{\parindent}{0pt}
\usepackage{parskip}

% Profundidad de la tabla de contenidos.
\setcounter{secnumdepth}{3}

% Usa el paquete minted para mostrar trozos de código.
% Pueden seleccionarse el lenguaje apropiado y el estilo del código.
\usepackage{minted}
\usemintedstyle{colorful}
\setminted{fontsize=\small}
\setminted[haskell]{linenos=false,fontsize=\small}
\renewcommand{\theFancyVerbLine}{\sffamily\textcolor[rgb]{0.5,0.5,1.0}{\oldstylenums{\arabic{FancyVerbLine}}}}

% Archivos de configuración.
\input{macros}  % En macros.tex se almacenan las opciones y comandos para escribir matemáticas.
\input{classicthesis-config} % En classicthesis-config.tex se almacenan las opciones propias de la plantilla.

% Color institucional UGR
% \definecolor{ugrColor}{HTML}{ed1c3e} % Versión clara.
\definecolor{ugrColor}{HTML}{c6474b}  % Usado en el título.
\definecolor{ugrColor2}{HTML}{c6474b} % Usado en las secciones.

% Datos de portada
\usepackage{titling} % Facilita los datos de la portada
\author{Ricardo Ruiz Fernández de Alba} 
\date{\today}
\title{Generación de datos sintética a partir de un \\  sistema IoT de detección
de vehículos utilizando \\ modelos Transformer con atención en Python}

% Portada
\include{titlepage}
\usepackage{wallpaper}
\usepackage[main=spanish]{babel}


\begin{document}

\ThisULCornerWallPaper{1}{ugrA4.pdf}
\maketitle
\tableofcontents


\chapter*{Resumen}

% Los artículos y libros incluidos en el archivo research.bib pueden
% citarse desde cualquier punto del texto usando ~\cite.

Nos basamos en el trabajo desarrollado en~\cite{vaswani_attention_2017}.

Occaecati expedita cumque est. Aut odit vel nobis praesentium dolorem
consequatur. Et cumque quia recusandae fugiat earum repellat
porro. Earum et tempora vel voluptas. At sed animi qui hic eaque



\ctparttext{
  \color{black}
  \begin{center}
    En este capítulo se presentarán las definiciones y resultados fundamentales
    de la teoría de probabilidad que servirán como base para el desarrollo
    posterior del trabajo.  
  \end{center}
}
\part{Fundamentos matemáticos elementales}

\chapter{Sección primera}

Estableceremos la teoría bajo la suposición de que hay un conjunto no vacío
$\Omega$, que representa todos los posibles resultados de un experimento aleatorio.
Definimos un \textit{evento} como cualquier subconjunto de $\Omega$.

\begin{definition}[$\sigma$-álgebra]
Sea $\mathcal{P}$ partes de $\Omega$. Llamamos $\sigma$-álgebra a cualquier $\mathcal{A} \subset \mathcal{P}$ que cumpla:

\begin{itemize}
\item $ \mathcal{A} \neq \emptyset$.
\item Si $A_1, A_2, \ldots, A_n \in \mathcal{A}$ entonces $\bigcup_{i=1}^n A_i \in \mathcal{A}$.
\item Si $A \in \mathcal{A}$ entonces $A^c \in \mathcal{A}$.
\end{itemize}
\end{definition}

A partir de las propiedades anteriores, deducimos que $\Omega \in \mathcal{A}$ 
y que $\mathcal{A}$ también es cerrado bajo intersecciones numerables.

\begin{definition}[Función de probabilidad]
$P: \mathcal{A} \rightarrow [0,1]$ es una función de probabilidad si satisface los tres axiomas:
\begin{enumerate}
  \item $P(A) \geq 0 \quad \forall A \in \mathcal{A}$.
  \item $P(\Omega) = 1$
  \item $P$ es $\sigma$-aditiva, es decir: dada $\{A_n \}_{n \in \mathbb{N}} \subset \mathcal{A}$ es una sucesión de conjuntos disjuntos dos a dos entonces
    $$
    P\left( \bigcup_{i\in \mathbb{N}} A_i \right) = \sum_{i \in \mathbb{N}} P(A_i).
    $$
\end{enumerate}
\end{definition}

Identificaremos como suceso seguro al evento que siempre ocurre. La primera
condición nos asegura que el suceso seguro tiene la probabilidad más alta
posible. La segunda condición garantiza que la probabilidad es no negativa.
Finalmente, la tercera condición establece que, para un conjunto de sucesos
disjuntos, la probabilidad de que ocurra alguno de ellos es igual a la suma de
las probabilidades individuales de cada suceso.

\newpage

\begin{proposition}
Toda medida de probabilidad $P$, cumple:
  \begin{itemize}
    \item $P(\emptyset) = 0$
    \item Dados $A, B \in \mathcal{A}$, entonces $P(A \cup B) = P(A) + P(B) - P(A \cap B)$.
  \end{itemize}
\end{proposition}

\begin{proof}
  La primera propiedad se deduce de la $\sigma$-aditividad de $P$ y de que
  $\emptyset = \bigcup_{i=1}^\infty \emptyset$. La segunda propiedad se deduce
  de la $\sigma$-aditividad de $P$ y de que $A \cup B = A \cup (B \setminus A)$ disjuntos.

\end{proof}

\begin{definition}[Espacio de probabilidad]
Definimos como \textit{espacio de medida} a la terna $(\Omega, \mathcal{A}, \mu)$, donde $\mu: \mathcal{A} \rightarrow R_0^+$ 
es una medida en $(\Omega, \mathcal{A})$ y \textit{espacio de probabilidad} a la tupla formada por $(\Omega, \mathcal{A}, P)$, 
donde $P$ es una medida de probabilidad en $(\Omega, \mathcal{A})$.
\end{definition}


\chapter{Sección segunda}

Sección segunda.

\ctparttext{\color{black}\begin{center}
Esta es una descripción de la parte de informática.
\end{center}}

\part{Parte de informática}
\chapter{Sección tercera}
El siguiente código es un ejemplo de coloreado de sintaxis e inclusión
directa de código fuente en el texto usando \texttt{minted}.

\begin{minted}[frame=lines]{haskell}
-- From the GHC.Base library.
class  Functor f  where
    fmap        :: (a -> b) -> f a -> f b

    -- | Replace all locations in the input with the same value.
    -- The default definition is @'fmap' . 'const'@, but this may be
    -- overridden with a more efficient version.
    (<$)        :: a -> f b -> f a
    (<$)        =  fmap . const

-- | A variant of '<*>' with the arguments reversed.
(<**>) :: Applicative f => f a -> f (a -> b) -> f b
(<**>) = liftA2 (\a f -> f a)
-- Don't use \$ here, see the note at the top of the page

-- | Lift a function to actions.
-- This function may be used as a value for `fmap` in a `Functor` instance.
liftA :: Applicative f => (a -> b) -> f a -> f b
liftA f a = pure f <*> a
-- Caution: since this may be used for `fmap`, we can't use the obvious
-- definition of liftA = fmap.

-- | Lift a ternary function to actions.
liftA3 :: Applicative f => (a -> b -> c -> d) -> f a -> f b -> f c -> f d
liftA3 f a b c = liftA2 f a b <*> c


{-# INLINABLE liftA #-}
{-# SPECIALISE liftA :: (a1->r) -> IO a1 -> IO r #-}
{-# SPECIALISE liftA :: (a1->r) -> Maybe a1 -> Maybe r #-}
{-# INLINABLE liftA3 #-}
{-# SPECIALISE liftA3 :: (a1->a2->a3->r) -> IO a1 -> IO a2 -> IO a3 -> IO r #-}
{-# SPECIALISE liftA3 :: (a1->a2->a3->r) ->
                                Maybe a1 -> Maybe a2 -> Maybe a3 -> Maybe r #-}

-- | The 'join' function is the conventional monad join operator. It
-- is used to remove one level of monadic structure, projecting its
-- bound argument into the outer level.
join              :: (Monad m) => m (m a) -> m a
join x            =  x >>= id
\end{minted}

Vivamus fringilla egestas nulla ac lobortis. Etiam viverra est risus,
in fermentum nibh euismod quis. Vivamus suscipit arcu sed quam dictum
suscipit. Maecenas pulvinar massa pulvinar fermentum
pellentesque. Morbi eleifend nec velit ut suscipit. Nam vitae
vestibulum dui, vel mollis dolor. Integer quis nibh sapien.



% Añade sección de referencias al final del documento.
% Selecciona un estilo de cita.
\bibliographystyle{alpha}
% En research.bib están las entradas de los artículos que citamos.
% Podemos cambiar el nombre del archivo aquí.
\bibliography{research}   

\end{document}