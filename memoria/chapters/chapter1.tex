\chapter{Introducción y objetivos}

\section{Introducción}

\subsection{Contextualización}

La Inteligencia Artificial (IA), impulsada por el avance en algoritmos de
aprendizaje automático ha experimentado un rápido desarrollo debido a su
reciente viabilidad de ejecución en hardware y la creciente disponibilidad de
datos. 

Sus fundamentos matemáticos, incluyendo el análisis, la probabilidad
y el álgebra lineal que conforman la teoría del aprendizaje estadístico
son esenciales para construir modelos que aprenden de los datos y realizan predicciones. 

Como aplicación particular se encuentra la capacidad de estos algoritmos para
identificar patrones en datos generados por las tecnologías de Internet
Industrial de las Cosas (IoT), basadas en redes de sensores y dispositivos
conectados a Internet que recopilan y transmiten datos en tiempo real.

\subsection{Descripción del problema}

El marco de este trabajo está relacionado con un proyecto de investigación de un sistema IoT
implementado en la región de Poqueira, en la Alpujarra granadina \cite{smartpoqueira}.
El sistema IoT ha procesado la información recogida por los sensores visuales
y la ha exportado a un conjunto de datos con múltiples características.

El entrenamiento de modelos de Aprendizaje Automático a partir de estos datos podría
predecir variables que mejorasen la gestión de la ocupación de la zona,
promoviendo así una mayor sostenibilidad en la comunidad local.

Sin embargo, la complejidad de la información obtenida, la presencia de ruido y
la falta de datos etiquetados, dificultan la tarea. En este contexto, la
generación de datos sintéticos se presenta como solución, pues 
permite añadir al conjunto de datos original muestras generadas artificialmente que
simulan las características de los datos reales. Esta
técnica facilita el entrenamiento de modelos más robustos y precisos, abordando
la escasez de etiquetas y mejorando la precisión de los modelos.

Aunque la literatura existente ha utilizado diversos Modelos de Generación Profunda (DGMs), 
incluyendo Redes Generativas Antagónicas (GANs) y Modelos de Autoencoders
Variacionales (VAEs) \cite{de_deep_2022}, en este proyecto nos centraremos en los Modelos
Transformers con atención, pues su capacidad de modelar dependencias a largo
plazo en los datos, los presenta como una alternativa prometedora.

\subsection{Estructura del trabajo}

Este trabajo se organiza en cinco partes, cada una de las cuales se divide en
capítulos que abordan diferentes aspectos del problema y su solución. La
estructura sigue las directrices establecidas para los trabajos de fin de grado.

\begin{itemize}
    \item \textbf{Primera parte:} Presenta la introducción y los objetivos del
    proyecto. 
    \item \textbf{Segunda parte:} Detalla los fundamentos matemáticos
    y teóricos del aprendizaje automático que sustentan el trabajo.
    
    \item \textbf{Tercera parte:} Se centra en el análisis de los modelos
    Transformer y perceptrones multicapa, explorando sus fundamentos,
    arquitecturas y aplicaciones en la generación de datos sintéticos.
    
    \item \textbf{Cuarta parte:} Describe el desarrollo del proyecto y los
    experimentos realizados, incluyendo detalles sobre el problema, el software
    utilizado, las métricas de evaluación y el preprocesamiento de datos.
    
    \item \textbf{Quinta parte:} Presenta las conclusiones y reflexiones finales
    sobre el trabajo, así como recomendaciones para futuras investigaciones.
    \end{itemize}

\subsection{Herramientas Matemáticas e Informáticas}

Para el desarrollo de este trabajo se utilizarán herramientas matemáticas e
informáticas clave. Los fundamentos matemáticos incluyen análisis
(diferenciabilidad y reglas de la cadena), probabilidad y estadística para el
manejo de la incertidumbre, y álgebra lineal, esenciales en el diseño y
optimización de modelos de \textit{machine learning}. 

La implementación se llevará a cabo en Python, utilizando bibliotecas como
\textit{Numpy}, \textit{Pandas}, y \textit{Matplotlib} para el análisis de
datos, y \textit{Sklearn} y \textit{Hugging Face} para el entrenamiento de los
modelos \textit{Transformers}.

\subsection{Bibliografía Fundamental}

\section{Objetivos}

\subsection{Objetivo General}

El objetivo general de este trabajo es analizar los fundamentos matemáticos del
aprendizaje automático y aplicar este conocimiento en la implementación de
modelos generativos, específicamente utilizando arquitecturas Transformer, para
la creación de datos sintéticos aplicables en la previsión de la ocupación a
partir del conjunto de datos.

\subsection{Objetivos Específicos}

\begin{enumerate}
    \item \textbf{Estudiar los fundamentos matemáticos del aprendizaje automático:}
    Revisar los conceptos matemáticos y teóricos del aprendizaje automático incluyendo
    teoría sobre cotas de generalización, optimización y regularización.   

    \item \textbf{Analizar las características de los datos recopilados:} Llevar
    a cabo un análisis del conjunto de datos obtenidos del sistema IoT de
    detección de vehículos, identificando limitaciones y patrones
    significativos.
    
    \item \textbf{Desarrollar modelos generativos:} Implementar modelos basados
    en perceptrones multicapa y arquitecturas de transformers con atención, con
    el fin de generar datos sintéticos representativos de los datos reales y que
    contribuyan a la mejora del análisis predictivo.
    
    \item \textbf{Analizar el rendimiento del modelo Transformer:} Evaluar el
    rendimiento y efectividad del modelo Transformer con atención a partir de
    métricas de evaluación comparación con sotros enfoques de modelado
    generativo. 

\end{enumerate}