\chapter{Introducción y objetivos}

\section{Introducción}

\subsection{Contextualización}
La generación de datos sintéticos ha emergido como una solución clave en el
campo del aprendizaje automático y la inteligencia artificial. En un entorno
donde la recopilación de datos reales puede ser complicada y costosa, esta
técnica ofrece la posibilidad de crear conjuntos de datos artificiales que
imitan patrones y características de datos reales. Esto resulta particularmente
relevante en aplicaciones donde los datos son escasos o donde existen
limitaciones éticas y legales para su obtención.

El sector turístico se presenta como un área propicia para la aplicación de la
generación de datos sintéticos. La capacidad de prever la ocupación hotelera se
convierte en una herramienta fundamental para la gestión eficiente de recursos.
Al utilizar datos recopilados de sistemas IoT de detección de vehículos, se
puede obtener información valiosa sobre la afluencia de turistas en distintas
áreas geográficas, permitiendo una mejor planificación en la oferta de servicios
turísticos.

\subsection{Descripción del problema}
A pesar de la potencialidad de los datos provenientes de sistemas IoT, su uso
presenta diversos desafíos. La heterogeneidad de los datos, la presencia de
ruido y la dificultad para modelar adecuadamente los patrones de comportamiento
de los turistas pueden afectar negativamente el rendimiento de los modelos de
aprendizaje automático. Por ello, se plantea la necesidad de desarrollar modelos
generativos que puedan crear datos sintéticos de alta calidad, reflejando las
características de los datos reales y optimizando así la capacidad predictiva de
los modelos.

\subsection{Estructura del trabajo}
Este trabajo se organiza en cinco partes bien diferenciadas, cada una de las
cuales se divide en un conjunto de capítulos que abordan diferentes aspectos del
problema y su solución. La estructura sigue las directrices establecidas para
los trabajos de fin de grado.

\begin{itemize}
    \item \textbf{Primera parte:} Presenta la introducción y los objetivos del
    proyecto. 
    \item \textbf{Segunda parte:} Detalla los fundamentos matemáticos
    y teóricos del aprendizaje automático que sustentan el trabajo. 
    \item \textbf{Tercera parte:} Se centra en el análisis de los modelos Transformer
    y perceptrones multicapa, explorando sus fundamentos, arquitecturas y
    aplicaciones en la generación de datos sintéticos.
    \item \textbf{Cuarta parte:} Describe el desarrollo del proyecto y los
    experimentos realizados, incluyendo detalles sobre el problema, el software
    utilizado, las métricas de evaluación y el preprocesamiento de datos. 
    \item \textbf{Quinta parte:} Presenta las conclusiones y reflexiones finales sobre
    el trabajo, así como recomendaciones para futuras investigaciones.
\end{itemize}

\subsection{Herramientas Matemáticas e Informáticas}
Para llevar a cabo este trabajo se emplearán diversas herramientas matemáticas e
informáticas. Las bases del aprendizaje automático, incluyendo algoritmos de
optimización y técnicas de regularización, se abordarán utilizando Python como
lenguaje de programación principal. Además, se hará uso de bibliotecas como NumPy,
Pandas, Matplotlib y Huggingface para el entrenamiento con Transformers.

\subsection{Bibliografía Fundamental}

\section{Objetivos}

\subsection{Objetivo General}
El objetivo general de este trabajo es analizar los fundamentos matemáticos del
aprendizaje automático y aplicar este conocimiento en la implementación de
modelos generativos, específicamente utilizando arquitecturas Transformer y
para la creación de datos sintéticos aplicables en la previsión de la ocupación a
través de los datos recopilados por un sistema IoT de detección de vehículos.

\subsection{Objetivos Específicos}
\begin{enumerate}
    \item \textbf{Estudiar las características de los datos recopilados:}
    Realizar un análisis exhaustivo de los datos obtenidos del sistema IoT de
    detección de vehículos, identificando sus limitaciones y patrones
    significativos. 
    \item \textbf{Implementar modelos generativos:} Desarrollar
    modelos basados en arquitecturas de transformers y perceptrones multicapa
    que permitan la creación de datos sintéticos representativos de los datos
    reales.
    \item \textbf{Evaluar la calidad de los datos generados:} Comparar los datos
    sintéticos generados con los datos reales para evaluar su calidad y utilidad
    en el entrenamiento de modelos de aprendizaje automático. 
    \item \textbf{Resolver el problema computacional:} Aplicar el modelo Transformer
    para abordar el problema de generación de datos sintéticos, analizando su
    rendimiento y efectividad en comparación con otros enfoques.
\end{enumerate}
